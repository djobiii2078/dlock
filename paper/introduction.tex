\section{Introduction}
\label{introduction}

\paragraph{Cloud computing and virtualization.} 

Users' computing needs become more diversified leading to a variety of workloads\cite{} forcing data centers holders to propose new classes of services to their clients\cite{}.
To efficiently use available hardware resources and power their different services, Cloud providers use virtualization\cite{}.
Virtualization provides a secure way to multiplex hardware resources between different users\cite{}.
With virtualization, users' workloads run in a blackbox entity refered as virtual machine (VM), which can be rapidly initialized and migrated from one server to another to ensure scalability and fault-tolerance. 

\paragraph{Energy usage and frequency states.}
Unfortunately, with Cloud computing becoming more attractive, the energy used by data centers around the world has skyrocketed in recent years\cite{} and reached up to $xx\%$ of the total use of energy on earth in $xxxx$.
This led to data centers being pointed out due to the potential effects on global warming\cite{}. 
To provide finer control on the energy used by a server, hardware vendors such as Intel and AMD introduced frequency states\cite{}.
Todays' processing core can have different frequency states characterized by the performance of the core in each state. 
For example, some Intel cores can enter \{P,E,C\}-states where P-states stand for performance --- favoring performance at the expense of 
