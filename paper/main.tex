%%%%%%%%%%%%%%%%%%%%%%%%%%%%%%%%%%%%%%%%%%%%%%%%%%%%%%%%%%%%%%%%%%%%%%%%%%%%%%%%
% Template for USENIX papers.
%
% History:
%
% - TEMPLATE for Usenix papers, specifically to meet requirements of
%   USENIX '05. originally a template for producing IEEE-format
%   articles using LaTeX. written by Matthew Ward, CS Department,
%   Worcester Polytechnic Institute. adapted by David Beazley for his
%   excellent SWIG paper in Proceedings, Tcl 96. turned into a
%   smartass generic template by De Clarke, with thanks to both the
%   above pioneers. Use at your own risk. Complaints to /dev/null.
%   Make it two column with no page numbering, default is 10 point.
%
% - Munged by Fred Douglis <douglis@research.att.com> 10/97 to
%   separate the .sty file from the LaTeX source template, so that
%   people can more easily include the .sty file into an existing
%   document. Also changed to more closely follow the style guidelines
%   as represented by the Word sample file.
%
% - Note that since 2010, USENIX does not require endnotes. If you
%   want foot of page notes, don't include the endnotes package in the
%   usepackage command, below.
% - This version uses the latex2e styles, not the very ancient 2.09
%   stuff.
%
% - Updated July 2018: Text block size changed from 6.5" to 7"
%
% - Updated Dec 2018 for ATC'19:
%
%   * Revised text to pass HotCRP's auto-formatting check, with
%     hotcrp.settings.submission_form.body_font_size=10pt, and
%     hotcrp.settings.submission_form.line_height=12pt
%
%   * Switched from \endnote-s to \footnote-s to match Usenix's policy.
%
%   * \section* => \begin{abstract} ... \end{abstract}
%
%   * Make template self-contained in terms of bibtex entires, to allow
%     this file to be compiled. (And changing refs style to 'plain'.)
%
%   * Make template self-contained in terms of figures, to
%     allow this file to be compiled. 
%
%   * Added packages for hyperref, embedding fonts, and improving
%     appearance.
%   
%   * Removed outdated text.
%
%%%%%%%%%%%%%%%%%%%%%%%%%%%%%%%%%%%%%%%%%%%%%%%%%%%%%%%%%%%%%%%%%%%%%%%%%%%%%%%%

\documentclass[sigplan,review,10pt]{acmart}
\renewcommand\footnotetextcopyrightpermission[1]{}

% to be able to draw some self-contained figs
\usepackage{tikz}
\usepackage{amsmath}
\usepackage{xspace}

% inlined bib file
\usepackage{filecontents}

%-------------------------------------------------------------------------------
\begin{filecontents}{\jobname.bib}
%-------------------------------------------------------------------------------
@Book{arpachiDusseau18:osbook,
  author =       {Arpaci-Dusseau, Remzi H. and Arpaci-Dusseau Andrea C.},
  title =        {Operating Systems: Three Easy Pieces},
  publisher =    {Arpaci-Dusseau Books, LLC},
  year =         2015,
  edition =      {1.00},
  note =         {\url{http://pages.cs.wisc.edu/~remzi/OSTEP/}}
}
@InProceedings{waldspurger02,
  author =       {Waldspurger, Carl A.},
  title =        {Memory resource management in {VMware ESX} server},
  booktitle =    {USENIX Symposium on Operating System Design and
                  Implementation (OSDI)},
  year =         2002,
  pages =        {181--194},
  note =         {\url{https://www.usenix.org/legacy/event/osdi02/tech/waldspurger/waldspurger.pdf}}}
\end{filecontents}

\newcommand{\solname}{\textsc{Dlock}\xspace}
\newcommand{\neurl}{\textsc{Neutral recombine}\xspace}
\newcommand{\neur}{\textsc{Neur}\xspace}

%-------------------------------------------------------------------------------
\begin{document}
%-------------------------------------------------------------------------------

%don't want date printed
\date{}

% make title bold and 14 pt font (Latex default is non-bold, 16 pt)
\title{\solname: Unlocking systems with the \neur approach}

%for single author (just remove % characters)
\author{
{\rm Blind submission}
% copy the following lines to add more authors
% \and
% {\rm Name}\\
%Name Institution
} % end author


%-------------------------------------------------------------------------------
\begin{abstract}
%-------------------------------------------------------------------------------
%Several systems maintain a shared state where several actors update the shared state (writers) and other actors 
%read (readers) from this shared state. 
%For example, a socket opened to several clients maintaining the number of active connections. 
%To ensure coherence between the different readers and writers, locking is essential. 
%Locking allows a writer to enter a critical section to update the shared state while other actors wait the completion 
%of its update. 
%Despite ensuring coherence and preventing inconsistency, locking comes with several shortcomings among which the waiting time
%of the different actors when one hold the lock. 

Several systems maintain a shared state between several writers and readers
where writers' execution flow is independent of the shared state, and the shared state 
update concern simple arithmetric operations.
For example, incrementing the number of read block requests in the Linux kernel. 

In this paper, we present \neurl (\neur) a novel mechanism to ensure consistency for the shared state without 
requiring locks or transactions. 
With \neur, instead of waiting for a lock holder, writers perform their arithmetic operation on the neutral element of the arithmetic operation and buffer the result.
Then, the different writers results are recombined to get the shared state which readers can retrieve. 
Compared to locking mechanims, \neur does not require waiting for a lock holder completion and is priority inversion free. 
%maybe reduces cache pollution
%should be evaluated
To ease \neur adoption and show its advantages, we built \solname.
\solname is a tool that parses the source code of 
a project and by feeding it with locks primitives definition, 
detects code sections using locks in a context that matches \texttt{NEUR} scope usage, and transforms the code 
to apply the \neur approach.
\solname also introduces a garbage collector to free writers local copy.

We apply \solname on crafted applications and real systems such as the Linux kernel, Zookeeper, HiBench, and Memcached.
Our evaluations show that the \solname generated versions achieve up to $xxx\times$ better throughput and execute up to $xxx\times$ faster 
for different configurations due to the \neur approach. 


%\textcolor{red}{Enchainer sur le problème d'opacités des VMs-unikernels ---- usecase chain of function and discussions about identifying idle tasks from processes in the kernel.}

\end{abstract}

\settopmatter{printfolios=true}
\maketitle
\pagestyle{plain}

\section{Introduction}
\label{introduction}

\paragraph{Cloud computing and virtualization.} 

Users' computing needs become more diversified leading to a variety of workloads\cite{} forcing data centers holders to propose new classes of services to their clients\cite{}.
To efficiently use available hardware resources and power their different services, Cloud providers use virtualization\cite{}.
Virtualization provides a secure way to multiplex hardware resources between different users\cite{}.
With virtualization, users' workloads run in a blackbox entity refered as virtual machine (VM), which can be rapidly initialized and migrated from one server to another to ensure scalability and fault-tolerance. 

\paragraph{Energy usage and frequency states.}
Unfortunately, with Cloud computing becoming more attractive, the energy used by data centers around the world has skyrocketed in recent years\cite{} and reached up to $xx\%$ of the total use of energy on earth in $xxxx$.
This led to data centers being pointed out due to the potential effects on global warming\cite{}. 
To provide finer control on the energy used by a server, hardware vendors such as Intel and AMD introduced frequency states\cite{}.
Todays' processing core can have different frequency states characterized by the performance of the core in each state. 
For example, some Intel cores can enter \{P,E,C\}-states where P-states stand for performance --- favoring performance at the expense of 

\section{Background and related work}
\label{background}

\subsection{Background}

\subsection{Related work}

Parler des transactions, lock-free, consensus numbers, java syncrhonization thread model, Rust forwarding lock 

\section{Problem: Incorrect accounting scenarios}
\label{motivations}
\section{Our solution: uiGOV}
\label{design}
\section{Evaluations}
\label{evaluations}
\section{Discussion and future work}
\label{discussion}
\section{Related work}
\label{rw}
NWAP\cite{10.1145/3466752.3480098} observes polling/interrupts to rapidly change P-core to meet SLO.
\textcolor{blue}{Also has refs on OS-driven DVFS}.

\cite{KARPOWICZ2018302} study of better DVFS models for applications. 

Peafowl\cite{10.1145/3419111.3421298} energy aware scheduling for key value stores. 

Yawn\cite{10.1145/3343737.3343740} idle state governor that uses online machine learning to improve idle state predictions on cores.
\section{Conclusion}
\label{conclusion}

%-------------------------------------------------------------------------------
\bibliographystyle{plain}
\bibliography{main}

%%%%%%%%%%%%%%%%%%%%%%%%%%%%%%%%%%%%%%%%%%%%%%%%%%%%%%%%%%%%%%%%%%%%%%%%%%%%%%%%
\end{document}
%%%%%%%%%%%%%%%%%%%%%%%%%%%%%%%%%%%%%%%%%%%%%%%%%%%%%%%%%%%%%%%%%%%%%%%%%%%%%%%%

%%  LocalWords:  endnotes includegraphics fread ptr nobj noindent
%%  LocalWords:  pdflatex acks